\documentclass[titlepage, a4paper]{article}
\usepackage[swedish]{babel}
\usepackage[utf8]{inputenc}
\usepackage{color}
\usepackage{graphicx}
\usepackage{etoolbox}
\usepackage{stringenc}
\usepackage{pdfescape}

% Sidformat
\usepackage{a4wide}

% Fixa Appendix-titlar
\usepackage[titletoc,title]{appendix}

% Bättre tabeller
\usepackage{tabularx}

% Bättre bildtexter
\usepackage[margin=10pt,font=small,labelfont=bf,labelsep=endash]{caption}

% Enkelt kommando som låter mig attgöra-markera text
\newcommand{\todo}[1] {\textbf{\textcolor{red}{#1}}}

% Nytt \paragraph låter oss ha onumrerade bitar
\makeatletter
\renewcommand\paragraph{\@startsection{paragraph}{4}{\z@}%
{-3.25ex\@plus -1ex \@minus -.2ex}%
{1.5ex \@plus .2ex}%
{\normalfont\normalsize\bfseries}}
\makeatother

\providecommand{\LIPSlogga}{../mall/logga1.png}
\providecommand{\LIPSdatum}{\today}

%% Headers och Footers
\usepackage{fancyhdr}
\pagestyle{fancy}
\lhead{\includegraphics[scale=0.4]{\LIPSlogga}}
\rhead{\ifdef{\LIPSutfardare}{Utfärdat av \LIPSutfardare \\\LIPSdatum}\LIPSdatum}
\lfoot{\LIPSkursnamn \\ \LIPSdokumenttyp}
\cfoot{\thepage}
\rfoot{\LIPSprojektgrupp \\ \LIPSprojektnamn}

%% Titelsida
\newcommand{\LIPSTitelsida}{%
{\ }\vspace{45mm}
\begin{center}
  \textbf{\Huge \LIPSdokument}
\end{center}
\begin{center}
  {\Large Grupp: \LIPSgrupp}
\end{center}
\begin{center}
  {\Large Redaktör: \LIPSredaktor}
\end{center}
\begin{center}
  {\Large \textbf{Version \LIPSversion}}
\end{center}
\vfill
\begin{center}
  {\large Status}\\[1.5ex]
  \begin{tabular}{|*{3}{p{40mm}|}}
    \hline
    Granskad & \LIPSgranskare & \LIPSgranskatdatum \\
    \hline
    Godkänd & \LIPSgodkannare & \LIPSgodkantdatum \\
    \hline
  \end{tabular}
\end{center}
\newpage
}

% Projektidentitet
\newenvironment{LIPSprojektidentitet}{%
{\ }\vspace{45mm}
\begin{center}
  {\Large PROJEKTIDENTITET}\\[0.5ex]
  {\small
  \LIPSartaltermin, \LIPSprojektgrupp\\
  Linköpings Tekniska Högskola, IDA
  }
\end{center}
\begin{center}
  {\normalsize Gruppdeltagare}\\
  \begin{tabular}{|l|l|p{25mm}|l|}
    \hline
    \textbf{Namn} & \textbf{Ansvar} & \textbf{Telefon} & \textbf{E-post} \\
    \hline
}%
{%
    \hline
  \end{tabular}
\end{center}
\begin{center}
  {\small
    \ifdef{\LIPSgruppadress}{\textbf{E-postlista för hela gruppen}: \LIPSgruppadress\\}{}
    \ifdef{\LIPSgrupphemsida}{\textbf{Hemsida}: \LIPSgrupphemsida\\[1ex]}{}
    \ifdef{\LIPSkund}{\textbf{Kund}: \LIPSkund\\}{}
    \ifdef{\LIPSkundkontakt}{\textbf{Kontaktperson hos kund}: \LIPSkundkontakt\\}{}
    \ifdef{\LIPSkursansvarig}{\textbf{Kursansvarig}: \LIPSkursansvarig\\}{}
    \ifdef{\LIPShandledare}{\textbf{Handledare}: \LIPShandledare\\}{}
  }
\end{center}
\newpage
}
\newcommand{\LIPSgruppmedlem}[4]{\hline {#1} & {#2} & {#3} & {#4} \\}

%% Dokumenthistorik
\newenvironment{LIPSdokumenthistorik}{%
\begin{center}
  Dokumenthistorik\\[1ex]
  %\begin{small}
    \begin{tabular}{|l|l|p{60mm}|l|l|}
      \hline
      \textbf{Version} & \textbf{Datum} & \textbf{Utförda förändringar} & \textbf{Utförda av} & \textbf{Granskad} \\
      }%
    {%
			\hline
    \end{tabular}
  %\end{small}
\end{center}
}

\newcommand{\LIPSversionsinfo}[5]{\hline {#1} & {#2} & {#3} & {#4} & {#5} \\}

% Kravlistor
\newenvironment{LIPSkravlista}{
	\center
		\tabularx{\textwidth}{| p{1.2cm} | p{1.9cm} | X | c |}
			\hline
			\textbf{Krav} & \textbf{Förändring} & \textbf{Beskrivning} & \textbf{Prioritet} \\\hline
}
{
		\endtabularx
	\endcenter
}

\newcounter{LIPSkravnummer}
\addtocounter{LIPSkravnummer}{1}
\newcommand{\LIPSkrav}[4][Krav \arabic{LIPSkravnummer}]{{#1} & {#2} & {#3} & {#4} \stepcounter{LIPSkravnummer}\\\hline}


% Leveranskravlistor
\newenvironment{LIPSleveranskravlista}{
	\center
		\tabularx{\textwidth}{| p{1.2cm} | p{1.9cm} | X | X |}
			\hline
			\textbf{Krav} & \textbf{Förändring} & \textbf{Beskrivning} & \textbf{Deadline}\\\hline
}
{
		\endtabularx
	\endcenter
}

\newcounter{LIPSleveranskravnummer}
\addtocounter{LIPSleveranskravnummer}{1}
\newcommand{\LIPSleveranskrav}[4][Krav \arabic{LIPSkravnummer}]{{#1} & {#2} & {#3} & {#4} \stepcounter{LIPSkravnummer}\\\hline}


% Milstolps-lista
\newenvironment{LIPSmilstolpar}{
	\center
		\tabularx{\textwidth}{| p{1.2cm} | X | l |}
			\hline
			\textbf{Nr} & \textbf{Beskrivning} & \textbf{Datum} \\\hline
}
{
		\endtabularx
	\endcenter
}

\newcounter{LIPSstolpnummer}
\addtocounter{LIPSstolpnummer}{1}
%\newcommand{\LIPSmilstolpe}[3][Krav \arabic{LIPSstolpnummer}]{{#1} & {#2} & {#3} \stepcounter{LIPSstolpnummer}\\\hline}
\newcommand{\LIPSmilstolpe}[3]{{#1} & {#2} & {#3} \\\hline}

% Aktivitets-lista
\newenvironment{LIPSaktivitetslista}{
	\center
		\tabularx{\textwidth}{| p{0.3cm} | X | c | c | c |}
			\hline
			\textbf{Nr} & \textbf{Beskrivning} & \textbf{Beroende av} & \textbf{Timmar} & \textbf{datum} \\\hline
}
{
		\endtabularx
	\endcenter
}

\newcounter{LIPSaktivitetsnummer}
\addtocounter{LIPSaktivitetsnummer}{1}
% \newcommand{\LIPSaktivitet}[4][\arabic{LIPSstolpnummer}]{{#1} & {#2} & {#3} & {#4} \stepcounter{LIPSstolpnummer}\\\hline}
\newcommand{\LIPSaktivitet}[5]{{#1} & {#2} & {#3} & {#4} & {#5}\\\hline}

% Mall för mötesprotokoll
\newenvironment{projektmote}[2]{
  {\ }\vspace{5mm}

  \centerline{\textbf{\Huge #1}}
  \vspace{2mm}
  \centerline{\LARGE #2}
  \vspace{10mm}

  \begin{itemize}
}
{
  \end{itemize}
}

\newcounter{paragrafnummer}
\addtocounter{paragrafnummer}{1}
\newcommand{\paragraf}[1]{\item{\textsection \arabic{paragrafnummer}. {#1}}\addtocounter{paragrafnummer}{1}}

% Mall för Statusrapport
\newenvironment{statusrapport}{
  \center
    \tabularx{\textwidth}{| p{0.4cm} | X | X | p{14.5mm} | p{13.5mm} | p{16.5mm} | p{16.5mm} |}
    \hline
    \textbf{Nr} & \textbf{Aktivitet} & \textbf{Beroenden} & \textbf{Planerad tid} & \textbf{Nedlagd tid} & \textbf{Planerad klar} & \textbf{Beräknat klart} \\\hline
}
{
    \endtabularx
  \endcenter
}

\newcommand{\aktivitetstatus}[7]{{#1} & {#2} & {#3} & {#4} & {#5} & {#6} & {#7} \\\hline}	% Importera generella layout-strukturer

% Information nödvändig för generella layout-strukturer
\newcommand{\LIPSgrupp}{50}
\newcommand{\LIPSredaktor}{Martin Söderén}
\newcommand{\LIPSversion}{0.1}
\newcommand{\LIPSdokument}{Användarmanual}
\newcommand{\LIPSdokumenttyp}{Användarmanual}
\newcommand{\LIPSgranskatdatum}{-}
\newcommand{\LIPSgranskare}{-}
\newcommand{\LIPSgodkannare}{-}
\newcommand{\LIPSgodkantdatum}{-}
\newcommand{\LIPSkursnamn}{TSEA83}
\newcommand{\LIPSprojektnamn}{PONG}
\newcommand{\LIPSprojektgrupp}{Grupp 50}
\newcommand{\LIPSartaltermin}{VT, 2016}
\newcommand{\LIPSgrupphemsida}{https://gitlab.ida.liu.se/oskjo581/tsea83}
\newcommand{\LIPSkund}{LIU}
\newcommand{\LIPSkundkontakt}{-}
\newcommand{\LIPSkursansvarig}{Anders Nilsson}
\newcommand{\LIPShandledare}{Carl Ingemarsson}

% Dokument-specifika paket
\usepackage{tabularx}
\usepackage{tikz}	
\usepackage{amsmath}
\usepackage{amsfonts}
\usepackage{algorithm}
\usepackage{algpseudocode}
\usepackage{float}
\usetikzlibrary{shapes, arrows}

\pagenumbering{roman}

\begin{document}

\LIPSTitelsida

\begin{LIPSprojektidentitet}
	\LIPSgruppmedlem{Martin Söderén}{Senior Hardware design engineer}{070 816 32 41}{marso329@student.liu.se}
	\LIPSgruppmedlem{Oskar Joelsson}{Junior Hardware design engineer}{076 185 17 17}{oskjo581@student.liu.se}
	\LIPSgruppmedlem{Jesper Jakobsson}{Hardware design intern}{070 673 25 10}{jesja947@student.liu.se}
\end{LIPSprojektidentitet}

%\newpage
%\tableofcontents	%Innehållsförteckning

\newpage

\begin{LIPSdokumenthistorik}
\LIPSversionsinfo{0.1}{2016-02-16}{Första utkast}{Grupp 50}{}
\end{LIPSdokumenthistorik}

\newpage
\pagenumbering{arabic}	%Påbörja sidnumrering
\section{Controlunit}
\subsection{FB-fält}
\begin{table}[H]
  \centering
  \begin{tabular}{|l|l|l|l|}
    \hline
    \textbf{Värde} & \textbf{Skriver till} & \textbf{beskrivning} & \textbf{Testat} \\ \hline
     001 & IR & Tar värdet på bussen och skriver det IR i control unit &- \\ \hline
  010 & PM & Tar värdet på bussen och skriver det till den adress som ASR pekar på &- \\ \hline
   011 & PC & Tar värdet på bussen och skriver det till PC &- \\ \hline
     101 & ASR & Tar värdet på bussen och skriver det till ASR &- \\ \hline
      Övriga & -- & Gör inget &- \\ \hline
  \end{tabular}
\end{table}
Måste kolla hur det blir med enable signalerna då de nollas i FB om de andra enheterna hinner läsa av bussen.


\subsection{TB-fält}
\begin{table}[H]
  \centering
  \begin{tabular}{|l|l|l|l|}
    \hline
    \textbf{Värde} & \textbf{Läser från} & \textbf{beskrivning} & \textbf{Testat} \\ \hline
     001 & IR & Tar värdet från IR och placerar det på bussen &- \\ \hline
  010 & PM & Tar värdet ASr pekar på i PM och placerar det på bussen &- \\ \hline
   011 & PC & Tar värdet i PC och placerar det på bussen &- \\ \hline
      Övriga & -- & Gör inget &- \\ \hline
  \end{tabular}
\end{table}


\section{BLOCK\_RAM}
\subsection{pm\_enable}
Vid pm\_enable=0 så läser/skriver PM från bussen.
\subsection{pm\_write}
Vid pm\_write=0 så läser pm från bussen och placerar det på den adress som ASR pekar på.
Vid pm\_write 01 så skriver pm den data som ASR pekar på till bussen. 


\section{Opkoder}
\begin{table}[H]
  \centering
  \begin{tabular}{|l|l|l|l|}
    \hline
    \textbf{Instruktion} & \textbf{Opkod}  \\ \hline
  LOAD GRx,M,ADR & 0000  \\ \hline
   STORE GRx,M,ADR & 0001 \\ \hline
    ADD GRx,M,ADR & 0010\\ \hline
     SUB GRx,M,ADR & 0011  \\ \hline
      AND GRx,M,ADR & 0100 \\ \hline
       LSR GRx,M,Y & 0101 \\ \hline
        BRA ADR & 0110 \\ \hline
         BNE ADR & 0111 \\ \hline
          HALT  & 1000 \\ \hline
          STOREV GRx,M,ADR &1001  \\ \hline
     		CMP GRx,M,ADR & 1010 \\ \hline
     		BGE ADR & 1011 \\ \hline
     		BEQ ADR & 1100  \\ \hline
     		IN GRx & 1101  \\ \hline
     		OUT GRx & 1110 \\ \hline
  \end{tabular}
  \caption{Opkoder}

\end{table}

\section{reset}
\begin{table}[H]
  \centering
  \begin{tabular}{|l|l|l|l|}
    \hline
    \textbf{Variable} & \textbf{Nollas under reset i}  \\ \hline
  IR & FB field process i control\_unit  \\ \hline
  
  pm\_enable & FB field process i control\_unit  \\ \hline
  \end{tabular}
  \caption{Opkoder}

\end{table}

\section{statusvippor}

Z=1 då resultatet är noll vid add,sub,and,lsr och cmp.\\
N= 1 vid sub då resultatet skulle bli negativt.\\
O=1 då en addition leder till overflow\\
C=1 då en addition leder till overflow

\end{document}
